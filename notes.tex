\subsection{Node Attacks}\label{node_attacks}

\subsubsection{Centrality-based targeted Attack}\label{cent_attack}

In this section, we will apply node removals by batches based on different centralities. 
The attack here consists of splitting each network into 100 equal batches and removing the top 1\% nodes of each batch successively.
The centralities are only computed once as an initial step for computational feasibility, as there are 20 networks and 12 centralities.  
The centralities used are:

\newline

\textbf{Betweenness Centrality} (BC) is a measure of the number of times a node lies on the shortest path between other nodes \cite{betweeness}.

%formula
\begin{equation}
    BC(v) = \sum_{s \neq v \neq t \in V} \frac{\sigma_{st}(v)}{\sigma_{st}}
\end{equation}

Where $\sigma_{st}$ is the total number of shortest paths from node $s$ to node $t$ 
and $\sigma_{st}(v)$ is the number of those paths that pass through $v$.

Betweenness centrality measures the influence of a node over the flow of information between nodes. 
In attack contexts, it is used to identify the node that, if removed, will cause the most disruption to the flow of information.
Many works have used betweenness centrality for this purpose, including PUT REFERENCE HERE, 
and concluded that it is the best centrality measure for this purpose.
This is why it's crucial to include it in our analysis, as it will serve as a baseline for comparison.

\newline

\textbf{Closeness Centrality} (CC) is a measure of how close a node is to all other nodes in the network.

%formula
\begin{equation}
    CC(v) = \frac{1}{\sum_{u \in V} d(v,u)}
\end{equation}

Where $d(v,u)$ is the shortest path distance between nodes $v$ and $u$.

Closeness centrality measures how fast information will spread from a given node to all other nodes in the network.
It depends on the distance between the node and all other nodes in the network,

\newline
\textbf{Degree Centrality} (DC) is a measure of the number of edges connected to a node.

%formula
\begin{equation}
    DC(v) = \sum_{u \in V} A_{vu}
\end{equation}

Where $A_{vu}$ is the element of the adjacency matrix $A$ at row $v$ and column $u$.

Degree centrality is the simplest centrality measure, and it is used to identify the most connected nodes in the network.
This can be particularly useful, where the most connected nodes have some kind of popularity,
and thus can spread information faster than other nodes. 
Degree and betweenness centralities most particularly have been used extensively in the literature studying attacks,
as they are proven to inflict the most damage to the network when targeted \cite{tomassini2023rewiring}.

\newline
\textbf{Complex Centrality} (PLCi) is a new metric introduced in 2021 \cite{contagion_paper}.
The paper handles the problem of social contagions which follows this theory: 
One node is not enough to spread information. 
Generally, "A peer must have contact with multiple activated peers for it to spread",
this is accounted for here, as the metric is based on the notion of \textit{wide bidges}.
A wide bridge describes an edge that connects 2 nodes in a way they will be able to spread information to each other.
The complex path length is determined by the number of sufficiently wide bridges traversed, 
allowing identification of chains of bridges and introducing a new measure of node centrality called complex centrality, 
defined as the average length of complex paths extending from a node. 
The node with the highest complex centrality in a graph is the one with the highest average complex path length.
The average complex path is the number of sufficient bridges
that are traversed in the complex path between neighbors of the source and the target itself.
This is a useful metric to study the spread of information in a network, 
its contagion-defined nature makes it a good candidate for targeted attacks in real-life networks,
where the spread of information is not as simple as a single node spreading to its neighbors, e.g., political opinions, social behaviors, transmitted viruses, etc.
Thus, in this paper, we will use this metric to study the effect of targeted attacks on the spread of information in a network,
knowing that nobody has used it for this purpose before.
To compute this metric, an R code was available on the paper's GitHub repository,
it was directed to account for the different networks we have, and the results were saved in CSV files that were accessed
in our project's main Python code to be used in the analysis.

\newline
\textbf{Percolation Centrality} (PC) 

\newline
\textbf{Mapping Entropy Degree Centrality} (MED) or Mapping Degree introduced in \cite{med} have been defined by the degree centrality
and the neighborhood of the node. It combines information about how connected a node is (degree centrality)
with the average amount of uncertainty (entropy) associated with the information in its local surroundings, 
providing a more comprehensive understanding of the node's significance in the network.
Having information on the node and its neighbors, it was proven to be more effective than traditional centralities
in identifying the most important nodes in a network. It is defined as follows:

%formula
\begin{equation}
    MED_k = - DC_k \times \sum_{i ∈ N_k} log(DC_i)
\end{equation}

Where $DC_k$ is the degree centrality of node $k$ and $N_k$ is the set of neighbors of node $k$.  
We will use this metric to study if it can provide a better attack strategy than the traditional degree centrality.

\newline
\textbf{Mapping Entropy Betweenness} (MEB), inspired by the mapping entropy centrality,
this one improves the betweenness centrality by basing on knowledge of the node and its neighbors \cite{meb}
This was introduced to account for the fact that degree centrality and betweenness centrality are two highly different properties,
specifically in this paper - studying terrorist networks- they pointed out that information spreading between terrorists depend on the node's
ability to act as a bridge between other nodes, rather than beng strictly depending on how many connections it has.
It was shown to perform well on 4 terrorist networks, better than the traditional betweenness centrality.


%formula
\begin{equation}
    MEB_k = - BC_k \times \sum_{i ∈ N_k} log(BC_i)
\end{equation}

Where $DC_k$ is the betweenness centrality of node $k$ and $N_k$ is the set of neighbors of node $k$.  
It's included among other centralities to study its attacks on a wider variety of networks and whether it can outperform the other centralities.

\newline
\textbf{Mapping Entropy Closeness} (MEC), inspired by the same Mapping entropy ideology.
This metric is special for our study where we would like to test if it can be better than its traditional counterpart,
having again centralities information from both each node and its neighbors.

%formula
\begin{equation}
    MEC_k = - CC_k \times \sum_{i ∈ N_k} log(CC_i)
\end{equation}

Where $CC_k$ is the closeness centrality of node $k$ and $N_k$ is the set of neighbors of node $k$. 

\newline
\textbf{Proximity centrality}, it's a way to measure how close a node is to the rest of the network \cite{proximity}
Also proposed on terrorist networks, proximity centrality provides a good way to identify "main conspirators",
who tend to be the closest to all nodes in the network. Identifying them would allow to access leaders
in a terrorism context.

%formula
\begin{equation}
    P_{Tc}(v) = \frac{\sum_{i=1}^{n} \sum_{j=1}^{n} | \text{spath}_{i,j} |}{\sum_{i=1}^{n} | \text{spath}_{vi} |}
\end{equation}


\newline
\textbf{} (DWT) is a newly established metric that ecaluates importance with respect to the number of connections and overlapping naeighbors

it starts by defining an edge strength $Sij$